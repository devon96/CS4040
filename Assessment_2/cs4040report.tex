\documentclass{csfourzero}

\title{CS4040: Impact of Meltdown \& Spectre}
\author{Konrad Dryja}
\date{\today}
% A useful package to support on-line references
\usepackage{url}
\usepackage{natbib}
\usepackage{xcolor}
\usepackage{listings}
\lstset{language=C,keywordstyle={\bfseries \color{blue}}}
\bibliographystyle{plain}
\abstract{The aftermath of discovery of side-channel attacks on the performance of modern chips.}


\begin{document}
\maketitle


\section{Introduction}
\label{sec:intro}
With speculative execution no longer considered safe, modern CPU manufacturers were forced to drop these features in favour of enhancing security. But at the same time, it resulted in sacrificing the performance of the chips. Probably the most harming aspect was that speculative execution was at the time an industry standard - after the processor designers hit a ceiling with potential clock speed, being hit by Moore's Law \cite{schaller1997moore}. I will not be focusing on the origin and details of the vulnerabilities, but rather the patches that followed - although will overview the basics in the next section.  

Personally, I found the topic the most interesting, as the aftermath is still haunting security researches to this day, since the fault was not the software, but inherent architecture was flawed. Moreover, frequently we are hearing news how the newest vulnerability based on side-channel execution has been discovered - with the definite fix being complete hardware replacement with a chip produced after 2017. This forced Intel, AMD, ARM to release very aggressive patches, greatly hurting the benchmarks.

This has also raised ethical questions - since some of the machines suffered as much as 50\& drop in performance. In the eyes of law, this could classify as false advertising followed by with many class-action lawsuits. As stated by Intel in their 2017 Annual report, as of February 2018, they were facing 30 customer faced suits along with two securities \cite{intelreport}. Intel perhaps is the company that was the most under fire, since Variant 1, also known as Meltdown, was mostly apparent in their chips 

\section{Background and related work}
\label{sec:lit}

When speaking about Spectre and Meltdown, it's very important to start from the very beginning - when in July 2017, a researcher Jann Horn from Google's Project Zero has discovered the vulnerability. Due to the severity and potential implications resulting from premature releasing of the findings, those were first communicated directly - on NDA basis - with manufacturers, hoping for an immediate fix. On January 2018, two papers were released by J. Horn et al. illustrating in-depth the vulnerability and how it could be replicated \cite{Lipp2018meltdown, Kocher2018spectre}. The papers present a throughout overview of the attack - the exploitation here is based on \textbf{Branch Prediction} (BP) along with \textbf{Out of Order Execution} (OOE). Those are optimizations techniques used by almost all CPUs on the market.

\begin{itemize}
  \item BP lets the processor "predict" the direction will go towards without explicitly evaluating the condition. For example, in a situation where \lstinline{if} statement was successful for 100 iterations, we can assume that 101st will be successful as well and thus prematurely execute the included code-block  

  \item OOE, on the other hand will often reorder scheduled operations leaving the most time-consuming actions till the end, executing the ones present in CPU cache immediately - assuming that those do not depend on each other, e.g., it's a simple summation.
\end{itemize}

Together they create a cheap and clever way to speed up the execution of binaries, but unbeknownst opened a pathway for side-channel attacks, exploiting the fact that CPU was executing the code that it wasn't meant to in the first place. 

Multiple patches have been created ever since to mitigate the negative effect, trying to minimize the impact on performance. M. L{\"o}w \cite{low2018overview} provides an overview of created patches and affected hardware.



\section{Research question}
\label{sec:rq}

Given the problem context (Section~\ref{sec:intro}) and background
(Section~\ref{sec:lit}), you should now be in a position to present
what you have investigated. \textbf{Pose this as a question.}

Then you should present your approach to addressing this
question.

Guide length: 500 words.

\section{Experimental Design}
\label{sec:exp}

What are your hypotheses? How are you going to test them? What is your
target population? What are your datasets; i.e.\ your sample of the
target population. What are the dependent and independent variables?

Guide length: 500 words.

% \section{Results}
% \label{sec:results}

% Present the results. A good way to organise this is via subsections
% for each hypothesis you tested. Include graphs of results
% (e.g.\ Figure~\ref{fig:data}), tests of significance, etc. If you have
% negative results, include them. A negative results is just as
% informative and useful as a positive one, sometimes more so.

% % \begin{figure}
% % \centerline{\includegraphics[width=5in]{basic-data-errors}}
% % \caption{Some results.}\label{fig:data}
% % \end{figure}

% Guide length: 500 words.

\section{Discussion}
\label{sec:discuss}

What do the results say? What have you learned from the
experiments? Have you identified a correlation between variables, or
causation? What are the limitations of what you've done? What further
experiments might be of benefit?

Guide length: 400 words.

\section{Conclusion}
\label{sec:conc}

What have you done and why? What have you shown through your
experiments?

Guide length: 100 words.

\bibliography{myrefs}

\end{document}
